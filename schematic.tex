%%%%%%%% ICML 2019 EXAMPLE LATEX SUBMISSION FILE %%%%%%%%%%%%%%%%%

\documentclass{article}

% Recommended, but optional, packages for figures and better typesetting:
\usepackage{microtype}
\usepackage{graphicx}
\usepackage{subcaption}
\usepackage{booktabs} % for professional tables
\usepackage{tikz}
\usepackage{listings}
\usepackage{courier}
\usepackage{amsmath, amsfonts}
\usepackage{algorithm,float}
\usepackage{lipsum}
\usepackage{xr}

\externaldocument{appendix}

\usetikzlibrary{automata,positioning}
\usetikzlibrary{decorations.pathreplacing}
\usetikzlibrary{decorations.pathmorphing}


\lstset{basicstyle=\footnotesize\ttfamily,breaklines=true}
\lstset{framextopmargin=50pt} % ,frame=bottomline}

% hyperref makes hyperlinks in the resulting PDF.
% If your build breaks (sometimes temporarily if a hyperlink spans a page)
% please comment out the following usepackage line and replace
% \usepackage{icml2019} with \usepackage[nohyperref]{icml2019} above.
\usepackage{hyperref}
%\usepackage{cleveref}

% Attempt to make hyperref and algorithmic work together better:
\newcommand{\theHalgorithm}{\arabic{algorithm}}
\newcommand{\setalglineno}[1]{%
  \setcounter{ALC@line}{\numexpr#1-1}}


\DeclareMathOperator{\GRU}{GRU}
\DeclareMathOperator{\BIGRU}{BI-GRU}
\DeclareMathOperator{\Cat}{categorical}
\DeclareMathOperator{\roll}{roll}
\DeclareMathOperator{\scan}{scan}
\DeclareMathOperator{\softmax}{softmax}
\DeclareMathOperator{\clip}{clip}
% Use the following line for the initial blind version submitted for review:
%\usepackage{icml2019}

% If accepted, instead use the following line for the camera-ready submission:
% The \icmltitle you define below is probably too long as a header.
% Therefore, a short form for the running title is supplied here:


\begin{document}



\begin{figure*}[t]
\resizebox{\textwidth}{!}{%
\begin{tikzpicture}[x=2mm,y=2mm,
    bin/.style={
% The shape:
rectangle,
% The size:
minimum height=8mm,
% The border:
thin, draw,
},
    weight/.style={
% The shape:
rectangle,
% The size:
minimum height=1mm,
% The border:
thin, draw,
},
    weights/.style={
% The shape:
rectangle,
% The size:
minimum height=1mm,
minimum width=24mm,
% The border:
thin, draw,
},
    memory/.style={
% The shape:
rectangle,
% The size:
minimum height=8mm,
minimum width=24mm,
% The border:
thin, draw,
},
]
  \node[memory,label={[xshift=-19.0mm]right:Memory $\mathbf{M}$}]
    (M) at (-3, 2) {};
  \node[memory]                         (P) at (5, 22) {$\mathbf{P}$};
  \node at (-7, -1) {$p_t$};
  \node[bin]                            (Mp) at (-7, 2) {};
  \node[minimum size=1cm, draw,circle]  (x) at (6, 2) {$\mathbf{x}_t$};
  \node[draw,circle,minimum size=1cm]   (GRU) at (0, 10) {GRU};
  \fill [black!5]                      (-1,16) rectangle (0.5,18);
  \fill [black!10]                      (0.5,16) rectangle (2,18);
  \fill [black!20]                      (2,16) rectangle (3.5,18);
  \fill [black!10]                      (3.5,16) rectangle (5,18);
  \fill [black!5]                      (5,16) rectangle (6.5,18);
  \draw [decorate,decoration={brace}]
  (-1,18.5) -- (11,18.5) node[midway,yshift=2mm]{\tiny softmax};
  \node[weights]                        (u) at (5, 17) {$\mathbf{u}_t$};
  \node                                 (p1) at (5, 30) {$d_{t}$};
  \node                                 (g) at (-5, 17) {$\mathbf{g}_t$};
  %\node                                 (cat) at (15, 30)
  %{$\Cat\left(\mathbf{P}\tilde{\mathbf{u}}_t\right)$};

  \draw [decorate,decoration={brace}]
  (-1,24.5) -- (11,24.5) node[midway,yshift=2mm]{};
  \node   (weighted-sum) at (5, 25.5) {\tiny weighted-sum};

  \draw [->] (Mp)          to [out=90,in=225] (GRU);
  \draw [->] (x)          to [out=90,in=315] (GRU);
  \draw [->] (GRU) to [out=135,in=270] node[left] {\tiny sample} (g);
  \draw [->] (GRU) to [out=45,in=270] (u);
  \draw [->] (GRU) edge[loop right] node {$\mathbf{h}_t$} (h);
  %\draw[->] (weighted-sum) -> (cat);
  \draw[->] (weighted-sum) -> node[right] {\tiny sample} (p1);

\end{tikzpicture}
}
\end{figure*}

\end{document}
