%%%%%%%% ICML 2019 EXAMPLE LATEX SUBMISSION FILE %%%%%%%%%%%%%%%%%

\documentclass{article}

% Recommended, but optional, packages for figures and better typesetting:
\usepackage{microtype}
\usepackage{graphicx}
\usepackage{subcaption}
\usepackage{booktabs} % for professional tables
\usepackage{tikz}
\usepackage{listings}
\usepackage{courier}
\usepackage{amsmath, amsfonts}
\usepackage{algorithm,float}
\usepackage{lipsum}
\usepackage{xr}

\externaldocument{appendix}

\usetikzlibrary{automata,positioning}
\usetikzlibrary{decorations.pathreplacing}
\usetikzlibrary{decorations.pathmorphing}


\lstset{basicstyle=\footnotesize\ttfamily,breaklines=true}
\lstset{framextopmargin=50pt} % ,frame=bottomline}

% hyperref makes hyperlinks in the resulting PDF.
% If your build breaks (sometimes temporarily if a hyperlink spans a page)
% please comment out the following usepackage line and replace
% \usepackage{icml2019} with \usepackage[nohyperref]{icml2019} above.
\usepackage{hyperref}
%\usepackage{cleveref}

% Attempt to make hyperref and algorithmic work together better:
\newcommand{\theHalgorithm}{\arabic{algorithm}}
\newcommand{\setalglineno}[1]{%
  \setcounter{ALC@line}{\numexpr#1-1}}


\DeclareMathOperator{\GRU}{GRU}
\DeclareMathOperator{\BIGRU}{BI-GRU}
\DeclareMathOperator{\Cat}{categorical}
\DeclareMathOperator{\roll}{roll}
\DeclareMathOperator{\scan}{scan}
\DeclareMathOperator{\softmax}{softmax}
\DeclareMathOperator{\clip}{clip}
% Use the following line for the initial blind version submitted for review:
%\usepackage{icml2019}

% If accepted, instead use the following line for the camera-ready submission:
% The \icmltitle you define below is probably too long as a header.
% Therefore, a short form for the running title is supplied here:


\begin{document}


\begin{figure*}[t]
\centering
  \begin{subfigure}{.3\textwidth}
    \centering
\begin{tikzpicture}[x=0.5mm,y=0.5mm,
    bit/.style={
% The shape:
rectangle,
% The size:
minimum height=.1mm,
minimum width=.1mm,
fill=black
},
    bin/.style={
% The shape:
rectangle,
% The size:
minimum height=8mm,
minimum width=3.5mm,
},
    matrix/.style={
% The shape:
rectangle,
% The size:
minimum height=8mm,
minimum width=28mm,
% The border:
thin, draw,
},
    vector/.style={
% The shape:
rectangle,
% The size:
minimum height=1mm,
minimum width=28mm,
% The border:
thin, draw,
},
]
  \draw [decorate,decoration={brace}]
  (-30,67) -- (-30,77) node[midway,left,inner sep=.1ex,text
    width=14mm,font=\tiny]{\tiny $\ell \times K$ Bernoulli distributions};
  \draw [decorate,decoration={brace}]
    (-30,84) -- (-30,105) node[midway,left,inner sep=.1ex,text
    width=14mm,font=\tiny]{example sampling from one of $\ell$
  edge distributions};
  \fill[black!70] (-28,67) rectangle (-24,69);
  \fill[black!60] (-24,67) rectangle (-20,69);
  \fill[black!10] (-20,67) rectangle (-16,69);
  \fill[black!20] (-16,67) rectangle (-12,69);
  \fill[black!40] (-12,67) rectangle (-8,69);
  \fill[black!60] (-8,67)  rectangle (-4,69);
  \fill[black!30] (-4,67)  rectangle (0,69);
  \fill[black!10] (0,67)   rectangle (4,69);
  \fill[black!15] (4,67)   rectangle (8,69);
  \fill[black!10] (8,67)  rectangle (12,69);
  \fill[black!10] (12,67) rectangle (16,69);
  \fill[black!20] (16,67) rectangle (20,69);
  \fill[black!30] (20,67) rectangle (24,69);
  \fill[black!10] (24,67) rectangle (28,69);

  \fill[black!10] (-28,69) rectangle (-24,71);
  \fill[black!00] (-24,69) rectangle (-20,71);
  \fill[black!10] (-20,69) rectangle (-16,71);
  \fill[black!10] (-16,69) rectangle (-12,71);
  \fill[black!10] (-12,69) rectangle (-8,71);
  \fill[black!40] (-8,69)  rectangle (-4,71);
  \fill[black!20] (-4,69)  rectangle (0,71);
  \fill[black!10] (0,69)   rectangle (4,71);
  \fill[black!50] (4,69)   rectangle (8,71);
  \fill[black!25] (8,69)   rectangle (12,71);
  \fill[black!10] (12,69)  rectangle (16,71);
  \fill[black!20] (16,69)  rectangle (20,71);
  \fill[black!30] (20,69)  rectangle (24,71);
  \fill[black!40] (24,69)  rectangle (28,71);

  \fill[black!30] (-28,71) rectangle (-24,73);
  \fill[black!40] (-24,71) rectangle (-20,73);
  \fill[black!30] (-20,71) rectangle (-16,73);
  \fill[black!20] (-16,71) rectangle (-12,73);
  \fill[black!10] (-12,71) rectangle (-8,73);
  \fill[black!10] (-8,71)  rectangle (-4,73);
  \fill[black!20] (-4,71)  rectangle (0,73);
  \fill[black!30] (0,71)   rectangle (4,73);
  \fill[black!60] (4,71)   rectangle (8,73);
  \fill[black!10] (8,71)   rectangle (12,73);
  \fill[black!20] (12,71)  rectangle (16,73);
  \fill[black!10] (16,71)  rectangle (20,73);
  \fill[black!20] (20,71)  rectangle (24,73);
  \fill[black!10] (24,71)  rectangle (28,73);

  \fill[black!40] (-28,73) rectangle (-24,75);
  \fill[black!30] (-24,73) rectangle (-20,75);
  \fill[black!20] (-20,73) rectangle (-16,75);
  \fill[black!10] (-16,73) rectangle (-12,75);
  \fill[black!10] (-12,73) rectangle (-8,75);
  \fill[black!30] (-8,73)  rectangle (-4,75);
  \fill[black!40] (-4,73)  rectangle (0,75);
  \fill[black!50] (0,73)   rectangle (4,75);
  \fill[black!45] (4,73)   rectangle (8,75);
  \fill[black!40] (8,73)   rectangle (12,75);
  \fill[black!30] (12,73)  rectangle (16,75);
  \fill[black!10] (16,73)  rectangle (20,75);
  \fill[black!10] (20,73)  rectangle (24,75);
  \fill[black!10] (24,73)  rectangle (28,75);

  \fill[black!10] (-24,75) rectangle (-20,77);
  \fill[black!30] (-20,75) rectangle (-16,77);
  \fill[black!40] (-16,75) rectangle (-12,77);
  \fill[black!80] (-12,75) rectangle (-8,77);
  \fill[black!30] (-8,75)  rectangle (-4,77);
  \fill[black!20] (-4,75)  rectangle (0,77);
  \fill[black!40] (0,75)   rectangle (4,77);
  \fill[black!15] (4,75)   rectangle (8,77);
  \fill[black!10] (8,75)   rectangle (12,77);
  \fill[black!10] (12,75)  rectangle (16,77);
  \fill[black!30] (16,75)  rectangle (20,77);
  \fill[black!60] (20,75)  rectangle (24,77);
  \fill[black!40] (24,75)  rectangle (28,77);

  \node[bin,opacity=.3,fill=cyan]   at (-24.7, 24) {};
  \node[bin,opacity=.3,fill=green]  at (-17.7, 24) {};
  \node[bin,opacity=.3,fill=yellow] at (-10.7, 24) {};
  \node[bin,opacity=.3,fill=orange] at (-3.7, 24)  {};
  \node[bin,opacity=.3,fill=red]    at (3.3, 24)   {};
  \node[bin,opacity=.3,fill=purple] at (10.3, 24)  {};
  \node[bin,opacity=.3,fill=violet] at (17.3, 24)  {};
  \node[bin,opacity=.3,fill=blue]   at (24.3, 24)  {};

  \node[bin,opacity=.3,fill=yellow] at (-24.7,48) {};
  \node[bin,opacity=.3,fill=orange] at (-17.7,48) {};
  \node[bin,opacity=.3,fill=red]    at (-10.7,48) {};
  \node[bin,opacity=.3,fill=purple] at (-3.7,48)  {};
  \node[bin,opacity=.3,fill=violet] at (3.3,48)   {};
  \node[bin,opacity=.3,fill=blue]   at (10.3,48)  {};
  \node[bin,opacity=.3,fill=cyan]   at (17.3,48)  {};
  \node[bin,opacity=.3,fill=green]  at (24.3,48)  {};

  %\draw[step=2mm] (-28.1,-4.1) grid (28.1,8.1);
  \node[matrix]                                 (I) at (0, 0)          {Instruction $\mathbf{I}$};
  %\node                                 () at (0, -10)         {Instruction $\mathbf{I}$};
  \node[matrix]                         (M) at (0, 24)         {$\mathbf{M}$};
  \node[matrix]                         (rolled) at (0, 48)    {$\mathbf{M}$};
  \node[vector]                         (B) at (0, 72)         {$\mathbf{B}$};
  \node[]                               (samples) at (0, 86) {};
  \draw[step=2mm] (-28.1,84) grid (28.1,88);

  \fill[black] (-8,84) rectangle (-12,88);
  \fill[black] (0,84) rectangle (4,88);
  \fill[black] (20,84) rectangle (24,88);

  \node[]                               (samples) at (0, 86) {};
  \draw[step=2mm] (-28.1,96) grid (28.1,100);
  \fill[black] (-8,96) rectangle (-12,100);
  \node[]                               (samples1) at (0, 86) {};
  \node[]                               (sample) at (0, 98) {};
  \node[]                               () at (-10, 104) {$d_t$};
  \draw[->] (-28,104) -> node[above] {\tiny choose first} (-14,104);


  \draw[->] (-28,58) -> (28,58);
  \draw[->] (28,60) -> (-28,60);

  \draw[->] (I)      -> node[right] {\tiny embed}             (M);
  \draw[->] (M)      -> node[right] {\tiny roll}              (rolled);
  \draw[->] (rolled) -> node[right,yshift=1mm] {\tiny Bidirectional GRU} (B);
  \draw[->] (B)      -> node[right] {\tiny sample}            (samples);
  %\draw[->] (samples1)      -> node[right] {\tiny sample}            (sample);
\end{tikzpicture}
\caption{\textbf{Encoding of instructions:} Architecture for
  mapping instruction $\mathbf{I}$ to pointer transition values $d_t$.  Note that $\mathbf{P}$ contains $\ell$ such distributions and
$\tilde{\mathbf{u}}_t$ chooses among them.}
  \label{schematic1}
  \end{subfigure}
\hfill
  \begin{subfigure}{.3\textwidth}
    \centering
\begin{tikzpicture}[x=2mm,y=2mm,
    bin/.style={
% The shape:
rectangle,
% The size:
minimum height=8mm,
% The border:
thin, draw,
},
    weight/.style={
% The shape:
rectangle,
% The size:
minimum height=1mm,
% The border:
thin, draw,
},
    weights/.style={
% The shape:
rectangle,
% The size:
minimum height=1mm,
minimum width=24mm,
% The border:
thin, draw,
},
    memory/.style={
% The shape:
rectangle,
% The size:
minimum height=8mm,
minimum width=24mm,
% The border:
thin, draw,
},
]
  \node[memory,label={[xshift=-19.0mm]right:Memory $\mathbf{M}$}]
    (M) at (-3, 2) {};
  \node[memory]                         (P) at (5, 22) {$\mathbf{P}$};
  \node at (-7, -1) {$p_t$};
  \node[bin]                            (Mp) at (-7, 2) {};
  \node[minimum size=1cm, draw,circle]  (x) at (6, 2) {$\mathbf{x}_t$};
  \node[draw,circle,minimum size=1cm]   (GRU) at (0, 10) {GRU};
  \fill [black!5]                      (-1,16) rectangle (0.5,18);
  \fill [black!10]                      (0.5,16) rectangle (2,18);
  \fill [black!20]                      (2,16) rectangle (3.5,18);
  \fill [black!10]                      (3.5,16) rectangle (5,18);
  \fill [black!5]                      (5,16) rectangle (6.5,18);
  \draw [decorate,decoration={brace}]
  (-1,18.5) -- (11,18.5) node[midway,yshift=2mm]{\tiny softmax};
  \node[weights]                        (u) at (5, 17) {$\mathbf{u}_t$};
  \node                                 (p1) at (5, 30) {$d_{t}$};
  \node                                 (g) at (-5, 17) {$\mathbf{g}_t$};
  %\node                                 (cat) at (15, 30)
  %{$\Cat\left(\mathbf{P}\tilde{\mathbf{u}}_t\right)$};

  \draw [decorate,decoration={brace}]
  (-1,24.5) -- (11,24.5) node[midway,yshift=2mm]{};
  \node   (weighted-sum) at (5, 25.5) {\tiny weighted-sum};

  \draw [->] (Mp)          to [out=90,in=225] (GRU);
  \draw [->] (x)          to [out=90,in=315] (GRU);
  \draw [->] (GRU) to [out=135,in=270] node[left] {\tiny sample} (g);
  \draw [->] (GRU) to [out=45,in=270] (u);
  \draw [->] (GRU) edge[loop right] node {$\mathbf{h}_t$} (h);
  %\draw[->] (weighted-sum) -> (cat);
  \draw[->] (weighted-sum) -> node[right] {\tiny sample} (p1);

\end{tikzpicture}
\caption{\textbf{Generation of subtask parameter $\mathbf{g}_t$ and pointer
  delta $d_t$:} This diagram depicts the flow of information every time-step
  from memory $\mathbf{M}$, pointer $p_t$, and observation $\mathbf{x}_t$ to 
subtask parameters $\mathbf{g}_t$ and pointer movements $d_t$. Note that $p_{t+1} = p_t + d_t$.}
\end{subfigure}
\hfill
\begin{subfigure}{.3\textwidth}
  \centering
\begin{tikzpicture}[x=2mm,y=2mm,
    bin/.style={
% The shape:
rectangle,
% The size:
minimum height=8mm,
% The border:
thin, draw,
},
    memory/.style={
% The shape:
rectangle,
% The size:
minimum height=8mm,
minimum width=24mm,
% The border:
thin, draw,
},
]
  \node[memory] (I) at (-5, -10) {$\mathbf{I}$};
  \node[memory] (M) at (-5, -4) {$\mathbf{M}$};
  \node[bin] (H) at (-5, 2) {$\mathbf{H}'$};
  \node[minimum size=1cm, draw,circle]  (x) at (6, 2) {$\mathbf{x}_t$};
  \node                                 (g) at (0, 17) {$\mathbf{g}_t$};
  \node[draw,circle,minimum size=1cm]   (GRU) at (0, 10) {GRU};
  \draw [->] (I)  -> node[right] {\tiny embed} (M);
  \draw [->] (M)  -> node[right] {\tiny Bidirectional GRU} (H);
  \draw [->] (H)  to [out=90,in=225] (GRU);
  \draw [->] (x)   to [out=90,in=315] (GRU);
  \draw [->] (GRU) -> node[right] {\tiny sample} (g);
  \draw [->] (GRU) edge[loop right] node {$\mathbf{h}_t$} (h);

\end{tikzpicture}
\caption{Diagram of \textbf{No pointer} baseline architecture.}
\label{no-pointer}
\end{subfigure}
\end{figure*}

\end{document}
